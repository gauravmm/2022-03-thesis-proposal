
\begin{figure}
    \centering
    \begin{tikzpicture}
        \begin{groupplot}[
            group style={
                group name=my plots,
                group size=4 by 1,
                ylabels at=edge left,
                yticklabels at=edge left,
                horizontal sep=18pt
            },
            axis on top,% ----
            width=1.in,
            height=1.in,
            scale only axis,
            enlargelimits=false,
            xmin=-2,
            xmax=2,
            ymin=-2,
            ymax=2,
            ]

        \nextgroupplot[title={Simulated}]
        \addplot[] graphics[xmin=-2,ymin=-2,xmax=2,ymax=2] {./figures/pendulum/true-stream.eps};
        \nextgroupplot[title={Learned $f$}]
        \addplot[] graphics[xmin=-2,ymin=-2,xmax=2,ymax=2] {./figures/pendulum/nn-stream.eps};
        \nextgroupplot[title={Learned $V$}]
        \addplot[] graphics[xmin=-2,ymin=-2,xmax=2,ymax=2] {./figures/pendulum/nn-lyapunov.eps};
        \nextgroupplot[axis equal image, axis lines=none, xtick=\empty, ytick=\empty]
        \addplotgraphicsnatural [xmin=-2, xmax=2, ymin=-2, ymax=2] {./figures/pendulum/nn-lyapunov-cmap.eps};
    \end{groupplot}
    \end{tikzpicture}
    \caption{Dynamics of a simple damped pendulum. From left to right: the dynamics as simulated from first principles, the dynamics model $f$ learned by our method, and the Lyapunov function $V$ learned by our method (under which $f$ is non-expansive).}
    \label{fig:pendulum_experiment}
\end{figure}{}
